\begin{frame}

\frametitle{Devices, Devices, Devices}

%\begin{columns}[T]

%\begin{column}[T]{.5\textwidth}

\begin{center}

\large \textbf{Character Devices}

\end{center}

\begin{itemize}

\item Read/write one character at a time.

\item 1 character = 1 byte.

\end{itemize}

%\end{column}

%\begin{column}[T]{.5\textwidth}

\vspace{\fill}

\begin{center}

\large \textbf{Block Devices}

\end{center}

\begin{itemize}

\item Read/write one block at a time.

\item Block size is typically a multiple of 512 bytes.

\end{itemize}

%\end{column}

%\end{columns}

\end{frame}



\begin{frame}

\frametitle{Sectors and Clusters}


\begin{itemize}

\item A "block" is a logical abstraction.

\item A "sector" is an actual minimum unit of storage on a typical physical device.

\item A "block" or "cluster" is a group of sectors.

\end{itemize}

\end{frame}
